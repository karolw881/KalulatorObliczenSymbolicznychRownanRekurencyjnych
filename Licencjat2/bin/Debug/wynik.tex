\documentclass[12pt]{article}
\usepackage{amsmath} 
\usepackage{tikz}
\usetikzlibrary{positioning,shadows,arrows}
\usepackage[polish]{babel} \usepackage{polski} \tikzset{fact/.style={rectangle, rounded corners=1mm, draw=black,fill=green!20,drop shadow,text centered, anchor=north, text=black},
       leaf/.style={circle, draw=none, draw=black,fill=yellow!20, circular drop shadow,text centered, anchor=north, text=black}}
\begin{document}
  \begin{center}  \LARGE\textsc{Równanie  Rekurencyjne}  \end{center}  
  \paragraph{Metoda: Szereg Formalny}    
 \begin{equation}{{a}}_{{n}} = \begin{cases}  {5}, &  \text{dla } n = 0 \\ {6}, &  \text{dla } n = 1 \\ {3}\cdot{{a}}_{{n-1}} + {4}\cdot{{a}}_{{n-2}}, &  \text{dla } n > 1 \\\end{cases}\end{equation}

\noindent Skupiamy sie na równaniu:
\begin{equation} {{x}}_{{n}} = {3}\cdot{{x}}_{{n-1}} + {4}\cdot{{x}}_{{n-2}}\end{equation}


\noindent Dokonujemy obliczeń symbolicznych na szeregu wykorzystując tylko zalezność rekurencyjną:
\begin{equation}
\begin{split}
f(x)  &= {{a}}_{{0}}\cdot{x}^ {{0}} + {{a}}_{{1}}\cdot{x}^ {{1}} + \sum_{{n} = {2}}^{{\infty}}{{a}}_{{n}}\cdot{x}^ {{n}}\\ 
     & = {{a}}_{{0}}\cdot{x}^ {{0}} + {{a}}_{{1}}\cdot{x}^ {{1}} + \sum_{{i} = {2}}^{{\infty}}({3}\cdot{{a}}_{{n-1}} + {4}\cdot{{a}}_{{n-2}})\cdot{x}^ {{n}}\\ 
     & = {{a}}_{{0}}\cdot{x}^ {{0}} + {{a}}_{{1}}\cdot{x}^ {{1}} + \sum_{{n} = {2}}^{{\infty}}({3}\cdot{{a}}_{{n-1}})\cdot{x}^ {{n}} + ({4}\cdot{{a}}_{{n-2}})\cdot{x}^ {{n}}\\ 
     & = {{a}}_{{0}}\cdot{x}^ {{0}} + {{a}}_{{1}}\cdot{x}^ {{1}} + \sum_{{n} = {2}}^{{\infty}}({3}\cdot({{a}}_{{n-1}}\cdot{x}^ {{n-1}}))\cdot{x} + \sum_{{n} = {2}}^{{\infty}}({4}\cdot({{a}}_{{n-2}}\cdot{x}^ {{n-2}}))\cdot{x}^ {{2}}\\ 
     & = {{a}}_{{0}}\cdot{x}^ {{0}} + {{a}}_{{1}}\cdot{x}^ {{1}} + ({3}\cdot{x})\cdot\sum_{{j} = {1}}^{{\infty}}{{a}}_{{j}}\cdot{x}^ {{j}} + ({4}\cdot{x}^ {{2}})\cdot\sum_{{i} = {0}}^{{\infty}}{{a}}_{{i}}\cdot{x}^ {{i }}\\ 
     & = {{a}}_{{0}}\cdot{x}^ {{0}} + {{a}}_{{1}}\cdot{x}^ {{1}} + {3}\cdot{x}\cdot(({f(x)}- ( {{a}}_{{0}} ) )) + {4}\cdot{x}^ {{2}}\cdot{f(x)}\\ 
\end{split}
\end{equation}
\noindent Stąd uzyskujemy równość:
\begin{equation}
{f(x)}- ( {3}\cdot{x}\cdot{f(x)} ) - ( {4}\cdot{x}^ {{2}}\cdot{f(x)} ) =  {{a}}_{{0}}\cdot{x}^ {{0}} + {{a}}_{{1}}\cdot{x}^ {{1}} + {3}\cdot{x}\cdot(- ( {{a}}_{{0}} ) )\end{equation}
Wyodrębniamy $f(x)$:
\begin{equation}
{f(x)}\cdot(({1}- ( {3}\cdot{x} ) - ( {4}\cdot{x}^ {{2}} ) )) = {{a}}_{{0}} + {x}\cdot(({{a}}_{{1}} + (- ( {3} ) )\cdot{{a}}_{{0}}\cdot{x}^ {{0}}))\end{equation}
Skąd ostatecznie $f(x)$:
\begin{equation}
f(x) =    \frac{{{a}}_{{0}} + {x}\cdot(({{a}}_{{1}} + (- ( {3} ) )\cdot{{a}}_{{0}}\cdot{x}^ {{0}}))}{{1}- ( {3}\cdot{x} ) - ( {4}\cdot{x}^ {{2}} ) }
\end{equation}
W kolejnym etapie rozkładamy ułamek na ułamki proste.
W tym celu odnajdujemy parametry pierwiastków $x_1$, $x_2$ dla których:\begin{equation}
{1}- ( {3}\cdot{x} ) - ( {4}\cdot{x}^ {{2}} ) = -{4}\cdot (x-x_1)\cdot(x-x_2)\end{equation}
Obliczamy delte:
$\delta=\left({-3}\right)^ {{2}} + (- ( {4} ) )\cdot(- ( {4} ) )\cdot{1}={9} + (- ( {4} ) )\cdot(- ( {4} ) )\cdot{1}={9} + {-4}\cdot(- ( {4} ) )\cdot{1}={9} + {-4}\cdot{-4}\cdot{1}={9} + {16}={25}={25}$
Stąd wyznaczamy pierwiastki $x_1$, $x_2$:\begin{center}\begin{minipage}{0.49\textwidth}
\begin{align}x_1 = & \frac{ - ( - ( {3} )  )  + \sqrt[{2}]{{25}} }{ {2}\cdot(- ( {4} ) ) } \notag \\ &= \frac{ - ( {-3} )  + \sqrt[{2}]{{25}} }{ {2}\cdot(- ( {4} ) ) } \notag \\ &= \frac{ {3} + \sqrt[{2}]{{25}} }{ {2}\cdot(- ( {4} ) ) } \notag \\ &= \frac{ {3} + {5}\cdot\sqrt[{2}]{{1}} }{ {2}\cdot(- ( {4} ) ) } \notag \\ &= \frac{ {3} + {5}\cdot{1} }{ {2}\cdot(- ( {4} ) ) } \notag \\ &= \frac{ {3} + {5} }{ {2}\cdot(- ( {4} ) ) } \notag \\ &= \frac{ {8} }{ {2}\cdot(- ( {4} ) ) } \notag \\ &= \frac{ {8} }{ {2}\cdot(- ( {4} ) ) } \notag \\ &= \frac{ {8} }{ {2}\cdot{-4} } \notag \\ &= \frac{ {8} }{ {-8} } \notag \\ &= \frac{ - ( {1} )  }{ {1} } \notag \\ &= \frac{ {-1} }{ {1} } \notag \\ &= {-1}\notag\end{align}\end{minipage}
\begin{minipage}{0.49\textwidth}
\begin{align}x_2 = & \frac{ - ( - ( {3} )  ) - ( \sqrt[{2}]{{25}} )  }{ {2}\cdot(- ( {4} ) ) } \notag \\ &= \frac{ - ( {-3} ) - ( \sqrt[{2}]{{25}} )  }{ {2}\cdot(- ( {4} ) ) } \notag \\ &= \frac{ {3}- ( \sqrt[{2}]{{25}} )  }{ {2}\cdot(- ( {4} ) ) } \notag \\ &= \frac{ {3}- ( {5}\cdot\sqrt[{2}]{{1}} )  }{ {2}\cdot(- ( {4} ) ) } \notag \\ &= \frac{ {3}- ( {5}\cdot{1} )  }{ {2}\cdot(- ( {4} ) ) } \notag \\ &= \frac{ {3}- ( {5} )  }{ {2}\cdot(- ( {4} ) ) } \notag \\ &= \frac{ {3} + {-5} }{ {2}\cdot(- ( {4} ) ) } \notag \\ &= \frac{ {-2} }{ {2}\cdot(- ( {4} ) ) } \notag \\ &= \frac{ {-2} }{ {2}\cdot(- ( {4} ) ) } \notag \\ &= \frac{ {-2} }{ {2}\cdot{-4} } \notag \\ &= \frac{ {-2} }{ {-8} } \notag \\ &= \frac{ - ( {-1} )  }{ {4} } \notag \\ &= \frac{ {1} }{ {4} } \notag \end{align}\end{minipage}
\end{center}
Z twierdzenia o rozkładzie ułamka wymiernego na ułamki proste, istnieją $C_1$, $C_2$ dla których:\begin{equation}
\frac{ {5} + {x}\cdot({6} + (- ( {3} ) )\cdot{5}\cdot{x}^ {{0}}) }{ {1}- ( {3}\cdot{x} ) - ( {4}\cdot{x}^ {{2}} )  }  = \frac{C_1}{x-x_1}+\frac{C_2}{x-x_2}\end{equation}
\begin{equation}
{5} + {x}\cdot({6}- ( {3}\cdot{5} ) ) = {x}\cdot((- ( {4} ) )\cdot{{C}}_{{1}} + (- ( {4} ) )\cdot{{C}}_{{2}}) + - ( (- ( {4} ) )\cdot{{C}}_{{1}}\cdot{{x}}_{{2}} ) - ( (- ( {4} ) )\cdot{{C}}_{{2}}\cdot{{x}}_{{1}} )  \end{equation} 
Wykorzystujac fakt, ze wielomiany są równe jeśli mają identyczne współczyniki uzyskujemy: \begin{equation}\left\{\begin{array}{r@{\;=\;}l}{5}& (- ( {4} ) )\cdot{{C}}_{{1}}\cdot{{x}}_{{2}} + (- ( {4} ) )\cdot{{C}}_{{2}}\cdot{{x}}_{{1}} \\ {6}- ( {3}\cdot{5} )  & (- ( {4} ) )\cdot{{C}}_{{1}} + (- ( {4} ) )\cdot{{C}}_{{2}} \end{array}\right.\end{equation}Podstawiamy wartości $x_1$, $x_2$ i robimy obliczenia pomocnicze:

$-{4} \cdot x_2 = (- ( {4} ) )\cdot\frac{ {1} }{ {4} }  = {-4}\cdot\frac{ {1} }{ {4} }  = - ( {4}\cdot\frac{ {1} }{ {4} }  ) 